\documentclass{article}
\usepackage[utf8]{inputenc}




\title{Homework $_$ 6 $_$ Proyecto $_$ final}
\author{Maykoll JAir Gil Martinez }
\date{}

\begin{document}
\maketitle
Como jugador de golf casi profesional siempre he tenido la duda para buscar la bola ideal para jugar, y una forma de hacer eso es con un fitting que es un modelo que analiza detalladamente todo lo relacionando con el jugador, pero para mi es mejor mucho mejor hacer modelo de bolas de golf y con matemáticas es la forma que yo pienso que me puede ayudar no solo a mi si no además jugadores que quieran ver que es lo que quieren que la bola haga si sale más alta o con más distancia, esto seda porque las bolas de golf hay dos tipos con hoyuelos hexágonos que hace que la bola salga más alta y hoyuelos circulares que la bola avance más.

Proporcionar un modelo matemático de la aerodinámica de la pelota de golf, donde influye el viento con respecto a la velocidad de la pelota donde interfiere los hoyuelos circulares o hexágonos de la pelota golf, donde influyen las propiedades físicas naturales como la gravedad, el peso, la densidad del fluido, la viscosidad, y la bola depende de su diámetro, la velocidad con que rota bola, el giro en términos de la velocidad periférica o ecuatorial, además el número de Reynolds hace participación, su comportamiento es lo que quiero modelar.

El enfoque que quiero tomar es probar si las bolas de golf con hoyuelos circulares son mejor que las hexagonales para el vuelo que van tomar las bolas, para así poder escoger una bola buena cuando se quiera jugar.

\textbf{Referencias}

Golf Ball Aerodynamics, of P W BEARMAN AND J K HARVEY of the  (Imperial College of Science and Technology)

A study of golf ball aerodynamic drag of Firoz Alam, Tom Steiner, Harum Chowdhury, Hazmin Moria, Iftekhar Khan, Fayez Aldawi, Aleksander Subic

Y por ultimo mientras tanto MAximun projectile range with drag and lift, whit particular applition to golf of Herman Erlichson,

\end{document}
